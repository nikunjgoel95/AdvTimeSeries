% Options for packages loaded elsewhere
\PassOptionsToPackage{unicode}{hyperref}
\PassOptionsToPackage{hyphens}{url}
%
\documentclass[
]{article}
\usepackage{lmodern}
\usepackage{amssymb,amsmath}
\usepackage{ifxetex,ifluatex}
\ifnum 0\ifxetex 1\fi\ifluatex 1\fi=0 % if pdftex
  \usepackage[T1]{fontenc}
  \usepackage[utf8]{inputenc}
  \usepackage{textcomp} % provide euro and other symbols
\else % if luatex or xetex
  \usepackage{unicode-math}
  \defaultfontfeatures{Scale=MatchLowercase}
  \defaultfontfeatures[\rmfamily]{Ligatures=TeX,Scale=1}
\fi
% Use upquote if available, for straight quotes in verbatim environments
\IfFileExists{upquote.sty}{\usepackage{upquote}}{}
\IfFileExists{microtype.sty}{% use microtype if available
  \usepackage[]{microtype}
  \UseMicrotypeSet[protrusion]{basicmath} % disable protrusion for tt fonts
}{}
\makeatletter
\@ifundefined{KOMAClassName}{% if non-KOMA class
  \IfFileExists{parskip.sty}{%
    \usepackage{parskip}
  }{% else
    \setlength{\parindent}{0pt}
    \setlength{\parskip}{6pt plus 2pt minus 1pt}}
}{% if KOMA class
  \KOMAoptions{parskip=half}}
\makeatother
\usepackage{xcolor}
\IfFileExists{xurl.sty}{\usepackage{xurl}}{} % add URL line breaks if available
\IfFileExists{bookmark.sty}{\usepackage{bookmark}}{\usepackage{hyperref}}
\hypersetup{
  pdftitle={assignment1},
  pdfauthor={nikunjgoel},
  hidelinks,
  pdfcreator={LaTeX via pandoc}}
\urlstyle{same} % disable monospaced font for URLs
\usepackage[margin=1in]{geometry}
\usepackage{color}
\usepackage{fancyvrb}
\newcommand{\VerbBar}{|}
\newcommand{\VERB}{\Verb[commandchars=\\\{\}]}
\DefineVerbatimEnvironment{Highlighting}{Verbatim}{commandchars=\\\{\}}
% Add ',fontsize=\small' for more characters per line
\usepackage{framed}
\definecolor{shadecolor}{RGB}{248,248,248}
\newenvironment{Shaded}{\begin{snugshade}}{\end{snugshade}}
\newcommand{\AlertTok}[1]{\textcolor[rgb]{0.94,0.16,0.16}{#1}}
\newcommand{\AnnotationTok}[1]{\textcolor[rgb]{0.56,0.35,0.01}{\textbf{\textit{#1}}}}
\newcommand{\AttributeTok}[1]{\textcolor[rgb]{0.77,0.63,0.00}{#1}}
\newcommand{\BaseNTok}[1]{\textcolor[rgb]{0.00,0.00,0.81}{#1}}
\newcommand{\BuiltInTok}[1]{#1}
\newcommand{\CharTok}[1]{\textcolor[rgb]{0.31,0.60,0.02}{#1}}
\newcommand{\CommentTok}[1]{\textcolor[rgb]{0.56,0.35,0.01}{\textit{#1}}}
\newcommand{\CommentVarTok}[1]{\textcolor[rgb]{0.56,0.35,0.01}{\textbf{\textit{#1}}}}
\newcommand{\ConstantTok}[1]{\textcolor[rgb]{0.00,0.00,0.00}{#1}}
\newcommand{\ControlFlowTok}[1]{\textcolor[rgb]{0.13,0.29,0.53}{\textbf{#1}}}
\newcommand{\DataTypeTok}[1]{\textcolor[rgb]{0.13,0.29,0.53}{#1}}
\newcommand{\DecValTok}[1]{\textcolor[rgb]{0.00,0.00,0.81}{#1}}
\newcommand{\DocumentationTok}[1]{\textcolor[rgb]{0.56,0.35,0.01}{\textbf{\textit{#1}}}}
\newcommand{\ErrorTok}[1]{\textcolor[rgb]{0.64,0.00,0.00}{\textbf{#1}}}
\newcommand{\ExtensionTok}[1]{#1}
\newcommand{\FloatTok}[1]{\textcolor[rgb]{0.00,0.00,0.81}{#1}}
\newcommand{\FunctionTok}[1]{\textcolor[rgb]{0.00,0.00,0.00}{#1}}
\newcommand{\ImportTok}[1]{#1}
\newcommand{\InformationTok}[1]{\textcolor[rgb]{0.56,0.35,0.01}{\textbf{\textit{#1}}}}
\newcommand{\KeywordTok}[1]{\textcolor[rgb]{0.13,0.29,0.53}{\textbf{#1}}}
\newcommand{\NormalTok}[1]{#1}
\newcommand{\OperatorTok}[1]{\textcolor[rgb]{0.81,0.36,0.00}{\textbf{#1}}}
\newcommand{\OtherTok}[1]{\textcolor[rgb]{0.56,0.35,0.01}{#1}}
\newcommand{\PreprocessorTok}[1]{\textcolor[rgb]{0.56,0.35,0.01}{\textit{#1}}}
\newcommand{\RegionMarkerTok}[1]{#1}
\newcommand{\SpecialCharTok}[1]{\textcolor[rgb]{0.00,0.00,0.00}{#1}}
\newcommand{\SpecialStringTok}[1]{\textcolor[rgb]{0.31,0.60,0.02}{#1}}
\newcommand{\StringTok}[1]{\textcolor[rgb]{0.31,0.60,0.02}{#1}}
\newcommand{\VariableTok}[1]{\textcolor[rgb]{0.00,0.00,0.00}{#1}}
\newcommand{\VerbatimStringTok}[1]{\textcolor[rgb]{0.31,0.60,0.02}{#1}}
\newcommand{\WarningTok}[1]{\textcolor[rgb]{0.56,0.35,0.01}{\textbf{\textit{#1}}}}
\usepackage{graphicx}
\makeatletter
\def\maxwidth{\ifdim\Gin@nat@width>\linewidth\linewidth\else\Gin@nat@width\fi}
\def\maxheight{\ifdim\Gin@nat@height>\textheight\textheight\else\Gin@nat@height\fi}
\makeatother
% Scale images if necessary, so that they will not overflow the page
% margins by default, and it is still possible to overwrite the defaults
% using explicit options in \includegraphics[width, height, ...]{}
\setkeys{Gin}{width=\maxwidth,height=\maxheight,keepaspectratio}
% Set default figure placement to htbp
\makeatletter
\def\fps@figure{htbp}
\makeatother
\setlength{\emergencystretch}{3em} % prevent overfull lines
\providecommand{\tightlist}{%
  \setlength{\itemsep}{0pt}\setlength{\parskip}{0pt}}
\setcounter{secnumdepth}{-\maxdimen} % remove section numbering

\title{assignment1}
\author{nikunjgoel}
\date{}

\begin{document}
\maketitle

\hypertarget{exercise-1}{%
\subsubsection{Exercise 1}\label{exercise-1}}

\textbf{Install R and RStudio (optional) now.}

I installed R and RStudio

\hypertarget{exercise-2}{%
\subsubsection{Exercise 2}\label{exercise-2}}

\textbf{Install xts and consequently zoo now.}

\begin{Shaded}
\begin{Highlighting}[]
\KeywordTok{library}\NormalTok{(xts)}
\end{Highlighting}
\end{Shaded}

\begin{verbatim}
## Loading required package: zoo
\end{verbatim}

\begin{verbatim}
## 
## Attaching package: 'zoo'
\end{verbatim}

\begin{verbatim}
## The following objects are masked from 'package:base':
## 
##     as.Date, as.Date.numeric
\end{verbatim}

\hypertarget{exercise-3}{%
\subsubsection{Exercise 3}\label{exercise-3}}

\textbf{Install the most recent version of astsa from Github.}

\begin{Shaded}
\begin{Highlighting}[]
\KeywordTok{library}\NormalTok{(astsa)}
\end{Highlighting}
\end{Shaded}

\hypertarget{exercise-4}{%
\subsubsection{Exercise 4}\label{exercise-4}}

\textbf{Explain what you get if you do this: ( 1 : 20 / 10 ) \%\% 1}

\begin{Shaded}
\begin{Highlighting}[]
\NormalTok{(}\DecValTok{1}\OperatorTok{:}\DecValTok{20}\OperatorTok{/}\DecValTok{10}\NormalTok{)}\OperatorTok{\%\%}\DecValTok{1}
\end{Highlighting}
\end{Shaded}

\begin{verbatim}
##  [1] 0.1 0.2 0.3 0.4 0.5 0.6 0.7 0.8 0.9 0.0 0.1 0.2 0.3 0.4 0.5 0.6 0.7 0.8 0.9
## [20] 0.0
\end{verbatim}

`1:20' Gave us 20 numbers starting with 1 and ending at 20 with an
increment of 1. `/10' Divided all the 20 numbers with 10 `\%\% 1' Gave
us a remainder of when dividing by 1.

\hypertarget{exercise-5}{%
\subsubsection{Exercise 5}\label{exercise-5}}

\textbf{Verify that 1/ i = − i where i = √ − 1 .}

\begin{Shaded}
\begin{Highlighting}[]
\DecValTok{1}\OperatorTok{/}\NormalTok{1i}
\end{Highlighting}
\end{Shaded}

\begin{verbatim}
## [1] 0-1i
\end{verbatim}

Answer is 0-1i which is of the form \(a + bi\) So b is -1.

\hypertarget{exercise-6}{%
\subsubsection{Exercise 6}\label{exercise-6}}

\begin{Shaded}
\begin{Highlighting}[]
\KeywordTok{exp}\NormalTok{(1i}\OperatorTok{*}\NormalTok{pi)}
\end{Highlighting}
\end{Shaded}

\begin{verbatim}
## [1] -1+0i
\end{verbatim}

\hypertarget{exercise-7}{%
\subsubsection{Exercise 7}\label{exercise-7}}

\textbf{Calculate these four numbers: cos ( π /2 ) , cos ( π ) , cos ( 3
π /2 ) , cos ( 2 π ) .}

\begin{Shaded}
\begin{Highlighting}[]
\NormalTok{cosseq \textless{}{-}}\StringTok{ }\NormalTok{( pi }\OperatorTok{*}\StringTok{ }\DecValTok{1} \OperatorTok{:}\StringTok{ }\DecValTok{4} \OperatorTok{/}\StringTok{ }\DecValTok{2}\NormalTok{ )}
\KeywordTok{cos}\NormalTok{(cosseq)}
\end{Highlighting}
\end{Shaded}

\begin{verbatim}
## [1]  6.123234e-17 -1.000000e+00 -1.836970e-16  1.000000e+00
\end{verbatim}

\hypertarget{exercise-8}{%
\subsubsection{Exercise 8}\label{exercise-8}}

\textbf{What is the difference between these two lines? 0 = x = y
0-\textgreater{} x-\textgreater{} y }

\begin{Shaded}
\begin{Highlighting}[]
\CommentTok{\#0=x=y}
\DecValTok{0}\NormalTok{{-}\textgreater{}x{-}\textgreater{}y}
\end{Highlighting}
\end{Shaded}

It gives an error on the first statement and second statement says
\(x = 0\) and \(y = 0\)

\hypertarget{exercise-9}{%
\subsubsection{Exercise 9}\label{exercise-9}}

\textbf{Why was y+z above the vector (10, 7, 4) and why is there a
warning?}

\begin{Shaded}
\begin{Highlighting}[]
\NormalTok{x =}\StringTok{ }\KeywordTok{c}\NormalTok{ ( }\DecValTok{1}\NormalTok{ , }\DecValTok{2}\NormalTok{ , }\DecValTok{3}\NormalTok{ , }\DecValTok{4}\NormalTok{ ); y =}\StringTok{ }\KeywordTok{c}\NormalTok{ ( }\DecValTok{2}\NormalTok{ , }\DecValTok{4}\NormalTok{ ); z =}\StringTok{ }\KeywordTok{c}\NormalTok{ ( }\DecValTok{8}\NormalTok{ , }\DecValTok{3}\NormalTok{ , }\DecValTok{2}\NormalTok{ )}
\NormalTok{y }\OperatorTok{+}\StringTok{ }\NormalTok{z}
\end{Highlighting}
\end{Shaded}

\begin{verbatim}
## Warning in y + z: longer object length is not a multiple of shorter object
## length
\end{verbatim}

\begin{verbatim}
## [1] 10  7  4
\end{verbatim}

Because y is just 2 elements and z is 3 elements. interpreter added 2 +
8 and 4+3 and left 2 as it is.

\hypertarget{exercise-10}{%
\subsubsection{Exercise 10}\label{exercise-10}}

\textbf{Create a directory that you will use for the course and use the
tricks previously mentioned to make it your working directory (or use
the default if you don't care). Load astsa and use help to find out
what's in the data file cpg . Write cpg as text to your working
directory.}

\begin{Shaded}
\begin{Highlighting}[]
\KeywordTok{library}\NormalTok{(astsa)}
\NormalTok{?cpg}
\KeywordTok{write}\NormalTok{(cpg,}\DataTypeTok{file =}\StringTok{\textquotesingle{}zzz.txt\textquotesingle{}}\NormalTok{,}\DataTypeTok{ncolumns=}\DecValTok{1}\NormalTok{)}
\end{Highlighting}
\end{Shaded}

\hypertarget{exercise-11}{%
\subsubsection{Exercise 11}\label{exercise-11}}

Find the file zzz.txt previously created (leave it there for now).
Solution: In RStudio , use the Files tab. Otherwise, go to your working
directory: getwd ()

\hypertarget{exercise-12}{%
\subsubsection{Exercise 12}\label{exercise-12}}

Scan and view the data in the file zzz.txt that you previously created.

\begin{Shaded}
\begin{Highlighting}[]
\NormalTok{(}\DataTypeTok{cost\_per\_gig =} \KeywordTok{scan}\NormalTok{(}\StringTok{"zzz.txt"}\NormalTok{))}
\end{Highlighting}
\end{Shaded}

\begin{verbatim}
##  [1] 2.13e+05 2.95e+05 2.60e+05 1.75e+05 1.60e+05 7.10e+04 6.00e+04 3.00e+04
##  [9] 3.60e+04 9.00e+03 7.00e+03 4.00e+03 2.00e+03 9.50e+02 8.65e+02 2.59e+02
## [17] 1.03e+02 6.29e+01 2.45e+01 1.25e+01 6.41e+00 2.68e+00 1.57e+00 1.38e+00
## [25] 6.70e-01 5.30e-01 4.20e-01 2.70e-01 7.00e-02
\end{verbatim}

\hypertarget{exercise-13}{%
\subsubsection{Exercise 13}\label{exercise-13}}

Make two vectors, say a with odd numbers and b with even numbers between
1 and 10. Then, use cbind to make a matrix, say x from a and b .
Afterthat, display each column of x separately.

\begin{Shaded}
\begin{Highlighting}[]
\NormalTok{a =}\StringTok{ }\KeywordTok{seq}\NormalTok{(}\DecValTok{1}\NormalTok{,}\DecValTok{10}\NormalTok{,}\DataTypeTok{by=}\DecValTok{2}\NormalTok{)}
\NormalTok{b =}\StringTok{ }\KeywordTok{seq}\NormalTok{(}\DecValTok{2}\NormalTok{,}\DecValTok{10}\NormalTok{,}\DataTypeTok{by=}\DecValTok{2}\NormalTok{)}
\NormalTok{x \textless{}{-}}\StringTok{ }\KeywordTok{cbind}\NormalTok{(a,b)}
\NormalTok{x[,}\DecValTok{1}\NormalTok{]}
\end{Highlighting}
\end{Shaded}

\begin{verbatim}
## [1] 1 3 5 7 9
\end{verbatim}

\begin{Shaded}
\begin{Highlighting}[]
\NormalTok{x[,}\DecValTok{2}\NormalTok{]}
\end{Highlighting}
\end{Shaded}

\begin{verbatim}
## [1]  2  4  6  8 10
\end{verbatim}

\hypertarget{exercise-14}{%
\subsubsection{Exercise 14}\label{exercise-14}}

Generate 100 standard normals and draw a boxplot of the results when
there are at least two displayed outliers (keep trying until you get
two).

\begin{Shaded}
\begin{Highlighting}[]
\KeywordTok{set.seed}\NormalTok{(}\DecValTok{911}\NormalTok{)}
\KeywordTok{boxplot}\NormalTok{(}\KeywordTok{rnorm}\NormalTok{(}\DecValTok{100}\NormalTok{))}
\end{Highlighting}
\end{Shaded}

\includegraphics{assignment1_files/figure-latex/unnamed-chunk-13-1.pdf}

\hypertarget{exercise-15}{%
\subsubsection{Exercise 15}\label{exercise-15}}

Write a simple function to return, for numbers x and y , the first input
raised to the power of the second input, and then use it to find the
square root of 25.

\begin{Shaded}
\begin{Highlighting}[]
\NormalTok{exercise15 \textless{}{-}}\StringTok{ }\ControlFlowTok{function}\NormalTok{(x,y)\{x}\OperatorTok{\^{}}\NormalTok{y\}}
\KeywordTok{exercise15}\NormalTok{(}\DecValTok{25}\NormalTok{,}\FloatTok{0.5}\NormalTok{)}
\end{Highlighting}
\end{Shaded}

\begin{verbatim}
## [1] 5
\end{verbatim}

\hypertarget{exercise-16}{%
\subsubsection{Exercise 16}\label{exercise-16}}

Add red horizontal and vertical dashed lines to the previously generated
graph to show that the fitted line goes through the point (x ,y ) .

\begin{Shaded}
\begin{Highlighting}[]
\KeywordTok{set.seed}\NormalTok{(}\DecValTok{666}\NormalTok{)}
\NormalTok{x =}\StringTok{ }\KeywordTok{rnorm}\NormalTok{(}\DecValTok{10}\NormalTok{)}
\NormalTok{y =}\StringTok{ }\DecValTok{1} \OperatorTok{+}\StringTok{ }\DecValTok{2}\OperatorTok{*}\NormalTok{x }\OperatorTok{+}\StringTok{ }\KeywordTok{rnorm}\NormalTok{(}\DecValTok{10}\NormalTok{)}
\KeywordTok{summary}\NormalTok{(fit \textless{}{-}}\StringTok{ }\KeywordTok{lm}\NormalTok{(y}\OperatorTok{\textasciitilde{}}\NormalTok{x))}
\end{Highlighting}
\end{Shaded}

\begin{verbatim}
## 
## Call:
## lm(formula = y ~ x)
## 
## Residuals:
##      Min       1Q   Median       3Q      Max 
## -1.22799 -1.01300  0.05943  0.60401  2.32480 
## 
## Coefficients:
##             Estimate Std. Error t value Pr(>|t|)    
## (Intercept)   1.0680     0.3741   2.855 0.021315 *  
## x             1.6778     0.2651   6.330 0.000225 ***
## ---
## Signif. codes:  0 '***' 0.001 '**' 0.01 '*' 0.05 '.' 0.1 ' ' 1
## 
## Residual standard error: 1.18 on 8 degrees of freedom
## Multiple R-squared:  0.8336, Adjusted R-squared:  0.8128 
## F-statistic: 40.07 on 1 and 8 DF,  p-value: 0.0002254
\end{verbatim}

\begin{Shaded}
\begin{Highlighting}[]
\KeywordTok{plot}\NormalTok{(x,y)}
\KeywordTok{abline}\NormalTok{(fit, }\DataTypeTok{col=}\DecValTok{4}\NormalTok{)}
\KeywordTok{abline}\NormalTok{ ( }\DataTypeTok{h =} \KeywordTok{mean}\NormalTok{ ( y ), }\DataTypeTok{col =} \DecValTok{2}\NormalTok{ , }\DataTypeTok{lty =} \DecValTok{2}\NormalTok{ )}
\KeywordTok{abline}\NormalTok{ ( }\DataTypeTok{v =} \KeywordTok{mean}\NormalTok{ ( x ), }\DataTypeTok{col =} \DecValTok{2}\NormalTok{ , }\DataTypeTok{lty =} \DecValTok{2}\NormalTok{ )}
\end{Highlighting}
\end{Shaded}

\includegraphics{assignment1_files/figure-latex/unnamed-chunk-15-1.pdf}

\end{document}
